\documentclass[a4paper,12pt]{article}
\usepackage[a4paper,left=3cm,right=2cm,top=2.5cm,bottom=2.5cm]{geometry}
\usepackage[utf8]{inputenc}
\usepackage{graphicx}
\usepackage[table]{xcolor}
\usepackage{array}
\usepackage{float}
\usepackage{enumitem}
\usepackage{hyperref}
\hypersetup{hidelinks}
\usepackage{amsmath}
\usepackage[ruled,vlined,linesnumbered,algoruled]{algorithm2e}
\usepackage{tabularx}

% Define subsubsubsection
\makeatletter
\newcommand\subsubsubsection{\@startsection{subsubsubsection}{4}{\z@}%
	{-3.25ex\@plus -1ex \@minus -.2ex}%
	{1.5ex \@plus .2ex}%
	{\normalfont\normalsize\bfseries}}
\newcommand*{\subsubsubsectionmark}[1]{}
\makeatother

% Dades de la pràctica
\newcommand{\titolPractica}{Supermaket Manager: \textbf{Manual d'Uusari}}
\newcommand{\identificadorEquip}{Subrgrup 11.3}
\newcommand{\PROPquatrimestre}{PROP - Quadrimestre de tardor 2024}
\newcommand{\versioLliurament}{Versió del lliurament 3.0}

% Dades de renovar comandes
\renewcommand*\contentsname{Continguts}
\renewcommand{\figurename}{Figura}
\renewcommand{\tablename}{Taula}

\begin{document}
	
	% Tapa del document
	\begin{titlepage}
		\begin{center}
			{\Large \textbf{\titolPractica}} \\[10cm]
			\textbf{\large \identificadorEquip} \\[1cm]
			Guillem Cabré Farré, \small{@guillem.cabre} \\
			Marc Peñalver Guilera, \small{@marc.penalver} \\
			Àlex Rodríguez Rodríguez, \small{@alex.rodriguez.r} \\
			Marc Teixidó Sala, \small{@marc.teixido} \\[2cm]
			\textbf{\versioLliurament} \\
			\textbf{\PROPquatrimestre} \\
			\textbf{Data: \today}
		\end{center}
	\end{titlepage}
	
	% Índex de continguts
	\tableofcontents
	\clearpage
	
	\section{Introducció}
	\subsection{Objectiu del manual}
	Aquest manual té com a finalitat guiar l'usuari en l'ús del programa \textbf{Supermarket Manager}, desenvolupat pel subgrup 11.3 en el marc del projecte PROP del quadrimestre de tardor del 2024. Proporciona instruccions detallades per configurar, operar i treure el màxim profit de l'aplicació.
	
	\subsection{Descripció general del programa}
	El \textbf{Supermarket Manager} és una aplicació dissenyada per optimitzar la gestió dels productes en un supermercat. El programa permet:
	\begin{itemize}
		\item Administrar el catàleg de productes, incloent-hi la seva importació i exportació des de fitxers JSON.
		\item Organitzar els productes a les prestatgeries de forma manual o mitjançant algorismes especialitzats.
		\item Visualitzar i gestionar la distribució dels productes, tenint en compte les seves restriccions de temperatura.
	\end{itemize}
	Aquest manual està dirigit a administradors i empleats del supermercat, amb o sense experiència prèvia en gestió informàtica.
	
	\section{Requeriments}
	
	\newpage
	\section{Interfície d'usuari}
	
	El programa informàtic té un total de 5 pantalles que permeten a l'usuari navegar entre diferents funcionalitats i modes d'operació. A continuació detallarem cada una d'elles.
	
	
	Aquí tens una versió millorada i més ben estructurada del teu text:
	
	\subsection{Log In}
	\label{sec:logIn}
	
	Aquesta finestra permet iniciar sessió tant a l'empleat com a l'administrador. Per defecte, la contrasenya d'ambdós perfils és la següent:
	
	\begin{itemize}
		\item \textbf{Administrador}: nom d'usuari \texttt{admin}, contrasenya \texttt{admin}.
		\item \textbf{Empleat}: nom d'usuari \texttt{employee}, contrasenya \texttt{employee}.
	\end{itemize}
	
	A la figura següent es pot observar la interfície de l'inici de sessió. Tot seguit, es detallen les funcionalitats disponibles.
	
	\begin{figure}[H] 
		\centering
		\includegraphics[width=0.75\linewidth]{assets/login.png}
		\caption{Interfície de Log In}
	\end{figure}
	
	\noindent Les funcionalitats de la vista són:
	
	\begin{enumerate}[itemsep=0pt, topsep=0pt]
		\item Camp d'entrada per introduir el nom d'usuari.
		\item Camp d'entrada per introduir la contrasenya.
		\item Botó de confirmació per validar les dades introduïdes. Navegarem a \textit{Main Screen} (\ref{sec:mainScreen}).
		\item Quadre de missatges per mostrar errors, com ara usuari inexistent o contrasenya incorrecta.
		\item Botó de tancament per sortir de l'aplicació o finalitzar la sessió activa.
	\end{enumerate}
	
	\newpage
	\subsection{Main Screen}
	\label{sec:mainScreen}
	
	\textbf{\textit{TODO: falta navegació cap a catàleg.}}
	
	Aquesta és la vista arrel, des de la que arribarem a totes les demès. La seva principal funcionalitat és mostrar la distribució del supermercat, a més d'oferir-nos les navegabilitats cap a les següents vistes. Hi ha funcionalitats d'aquestes vista que només li apareixeran a l'administrador. Quan enumerem les funcionalitats, mencionarem quines son només de l'administrador amb el símbol (\textbf{ADMIN}).
	
	\begin{figure}[H] 
		\centering
		\includegraphics[width=0.75\linewidth]{assets/mainscreen.png}
		\caption{Interfície de Main Screen}
	\end{figure}
	
	\noindent Les funcionalitats de la vista són:
	
	\begin{enumerate}[itemsep=0pt, topsep=0pt]
		\item Distribució dels productes en 3 prestatgeries contingents.
		\item Navegar cap a la següent prestatgeria cap a l'esquerra.
		\item Navegar cap a la següent prestatgeria cap a la dreta.
		\item (\textbf{ADMIN}) Botó que permet guardar la configuració en local.
		\item (\textbf{ADMIN}) Botó per guardar la configuració en un fitxer extern.
		\item (\textbf{ADMIN}) Botó per configurar la distribució del supermercat, navegarem a Edit Ditribution (\ref{sec:editDistribution}).
		\item Botó per o bé tancar la app o bé per tancar sessió, que ens farà navegar a Log In (\ref{sec:logIn}).
	\end{enumerate}
	
	\newpage
	\subsection{Edit Distribution}
	\label{sec:editDistribution}
	
	La vista Edit Distribution, tal i com indica el seu nom ens permetrà modificar la distribució de productes en el supermercat. Les operacions que pot fer seran detallades a continuació. \\
	
	Tenir en compte que aquesta vista només podrà ser accedida per l'\textbf{administrador} del sistema.
	
	\begin{figure}[H] 
		\centering
		\includegraphics[width=0.75\linewidth]{assets/editdistribution.png}
		\caption{Interfície de Edit Distribution}
	\end{figure}
	
	\noindent Les funcionalitats de la vista són:
	
	\begin{enumerate}[itemsep=0pt, topsep=0pt]
		\item Distribució dels productes en 3 prestatgeries contingents.
		\item Navegar cap a la següent prestatgeria cap a l'esquerra.
		\item Navegar cap a la següent prestatgeria cap a la dreta.
		\item Editar la prestatgeria, això ens portarà a Edit Shelving Unit (\ref{sec:editShelvingUnit}).
		\item Esborrar la prestatgeria.
		\item Afegir una prestatgeria en aquella posició. Aquest botó apareix dinàmicament al apropar el cursor. Està situat entre totes les prestatgeries, tot i que a la imatge només es veu una sola vegada.
		\item Botó que permet guardar la configuració en local.
		\item Botó per guardar la configuració en un fitxer extern.
		\item Botó per importar una configuració externa.
		\item Botó per crear des de zero una nova configuració.
		\item Botó per anar a Main Screen (\ref{sec:mainScreen}).
		\item Botó per o bé tancar la app o bé per tancar sessió, que ens farà navegar a Log In (\ref{sec:logIn}).
		\item Botó per ordenar els productes, amb un dels tres mètodes disponibles (GREEDY, APROXIMATION i BACKTRACKING).
		\item Botó per intercanviar la posició de dos productes o de dos prestatgeries.
	\end{enumerate}
	
	\newpage
	\subsection{Edit Shelving Unit}
	\label{sec:editShelvingUnit}
	
	De la anterior visita naveguem a aquesta. La vista Edit Shelving Unit és la encarregada de facilitar la edició d'una prestatgeria. Com la anterior, aquesta vista només serà accesible per l'\textbf{administrador}.
	
	\begin{figure}[H] 
		\centering
		\includegraphics[width=0.75\linewidth]{assets/editshelvingunit.png}
		\caption{Interfície de Edit Shelving Unit}
	\end{figure}
	
	\noindent Les funcionalitats de la vista són:
	
	\begin{enumerate}[itemsep=0pt, topsep=0pt]
		\item Permet modificar el tipus de temperatura, per acceptar els canvis, premer el tick; per cancelar-los, la creu.
		\item Botó per buidar la prestatgeria.
		\item Botó per eliminar la prestatgeria, a continuació navegarem a Edit\
	\end{enumerate}
	
	\newpage
	\subsection{Catalog}
	
	
	\newpage
	\section{Funcionalitats del programa}
	

\end{document}